%% start of file `template.tex'.
%% Copyright 2006-2015 Xavier Danaux (xdanaux@gmail.com), 2020-2021 moderncv maintainers (github.com/moderncv).
%
% This work may be distributed and/or modified under the
% conditions of the LaTeX Project Public License version 1.3c,
% available at http://www.latex-project.org/lppl/.


\documentclass[11pt,a4paper,sans]{moderncv}        % possible options include font size ('10pt', '11pt' and '12pt'), paper size ('a4paper', 'letterpaper', 'a5paper', 'legalpaper', 'executivepaper' and 'landscape') and font family ('sans' and 'roman')

% moderncv themes
\moderncvstyle{classic}                             % style options are 'casual' (default), 'classic', 'banking', 'oldstyle' and 'fancy'
\moderncvcolor{black}                               % color options 'black', 'blue' (default), 'burgundy', 'green', 'grey', 'orange', 'purple' and 'red'
%\renewcommand{\familydefault}{\sfdefault}         % to set the default font; use '\sfdefault' for the default sans serif font, '\rmdefault' for the default roman one, or any tex font name
%\nopagenumbers{}                                  % uncomment to suppress automatic page numbering for CVs longer than one page

% character encoding
%\usepackage[utf8]{inputenc}                       % if you are not using xelatex ou lualatex, replace by the encoding you are using
%\usepackage{CJKutf8}                              % if you need to use CJK to typeset your resume in Chinese, Japanese or Korean

% adjust the page margins
\usepackage[scale=0.75]{geometry}
\setlength{\footskip}{122.40004pt}                 % depending on the amount of information in the footer, you need to change this value. comment this line out and set it to the size given in the warning
%\setlength{\hintscolumnwidth}{3cm}                % if you want to change the width of the column with the dates
%\setlength{\makecvheadnamewidth}{10cm}            % for the 'classic' style, if you want to force the width allocated to your name and avoid line breaks. be careful though, the length is normally calculated to avoid any overlap with your personal info; use this at your own typographical risks...

% personal data
\name{Luke}{Conaboy}
% \title{Résumé title}                               % optional, remove / comment the line if not wanted
% \born{}                                 % optional, remove / comment the line if not wanted
\address{Centre for Astronomy and Particle Theory, University of Nottingham}{NG7 2RD}{UK}% optional, remove / comment the line if not wanted; the "postcode city" and "country" arguments can be omitted or provided empty
% \phone[mobile]{+1~(234)~567~890}                   % optional, remove / comment the line if not wanted; the optional "type" of the phone can be "mobile" (default), "fixed" or "fax"
% \phone[fixed]{+2~(345)~678~901}
% \phone[fax]{+3~(456)~789~012}
\email{luke.conaboy@nottingham.ac.uk}                               % optional, remove / comment the line if not wanted
\homepage{lconaboy.github.io}

% Social icons
%% \social[linkedin]{john.doe}                        % optional, remove / comment the line if not wanted
%% \social[xing]{john_doe}                           % optional, remove / comment the line if not wanted
%% \social[twitter]{jdoe}                             % optional, remove / comment the line if not wanted
\social[github]{lconaboy}                              % optional, remove / comment the line if not wanted
\social[orcid]{0000-0002-6580-7177}                  % optional, remove / comment the line if not wanted
%% \social[gitlab]{jdoe}                              % optional, remove / comment the line if not wanted
%% \social[stackoverflow]{0000000/johndoe}            % optional, remove / comment the line if not wanted
%% \social[bitbucket]{jdoe}                           % optional, remove / comment the line if not wanted
%% \social[skype]{jdoe}                               % optional, remove / comment the line if not wanted
%% \social[researchgate]{jdoe}                        % optional, remove / comment the line if not wanted
%% \social[researcherid]{jdoe}                        % optional, remove / comment the line if not wanted
%% \social[telegram]{jdoe}                            % optional, remove / comment the line if not wanted
%% \social[whatsapp]{12345678901}                     % optional, remove / comment the line if not wanted
%% \social[signal]{12345678901}                       % optional, remove / comment the line if not wanted
%% \social[matrix]{@johndoe:matrix.org}               % optional, remove / comment the line if not wanted
%% \social[googlescholar]{googlescholarid}            % optional, remove / comment the line if not wanted


% \extrainfo{additional information}                 % optional, remove / comment the line if not wanted
% \photo[64pt][0.4pt]{picture}                       % optional, remove / comment the line if not wanted; '64pt' is the height the picture must be resized to, 0.4pt is the thickness of the frame around it (put it to 0pt for no frame) and 'picture' is the name of the picture file
% \quote{Some quote}                                 % optional, remove / comment the line if not wanted

% bibliography adjustments (only useful if you make citations in your resume, or print a list of publications using BibTeX)
%   to show numerical labels in the bibliography (default is to show no labels)
%\makeatletter\renewcommand*{\bibliographyitemlabel}{\@biblabel{\arabic{enumiv}}}\makeatother
\renewcommand*{\bibliographyitemlabel}{[\arabic{enumiv}]}
%   to redefine the bibliography heading string ("Publications")
%\renewcommand{\refname}{Articles}

% bibliography with mutiple entries
%\usepackage{multibib}
%\newcites{book,misc}{{Books},{Others}}
%----------------------------------------------------------------------------------
%            content
%----------------------------------------------------------------------------------
\begin{document}
%\begin{CJK*}{UTF8}{gbsn}                          % to typeset your resume in Chinese using CJK
%-----       resume       ---------------------------------------------------------
\makecvtitle

\section{Employment}
\cventry{2023--present}{\normalfont Research Fellow}{\normalfont
  University of Nottingham}{Nottingham, UK}{}{Advisor: Prof James
  Bolton}

\cventry{2022--2023}{\normalfont Research Associate}{\normalfont
  University of Nottingham}{Nottingham, UK}{}{Advisor: Dr Emma
  Chapman}
\section{Education}
\cventry{2018--2022}{\normalfont PhD Astronomy}{\normalfont University of Sussex}{Brighton,
  UK}{}{Supervisor: Prof Ilian Iliev \\ Thesis: Simulations of structure formation and feedback at high redshift}  % arguments 3 to 6 can be left empty
\cventry{2014--2018}{\normalfont MSci Physics with Astronomy}{\normalfont University of
  Nottingham}{Nottingham, UK}{}{Degree classification: First}

\section{Publications}
\cventry{}{\normalfont I am an author on seven peer-reviewed articles
  (two as first author) on astrophysics ($h$-$\mathsf{index}=5$).}{}{}{}{}

\subsection{\em First author}
\cventry{2025}{\normalfont{\em The connection between high-redshift galaxies and Lyman $\alpha$ transmission in the Sherwood-Relics simulations of patchy reionisation}}{}{\textbf{Conaboy}, Bolton, Keating, Haehnelt, Kulkarni, Puchwein}{MNRAS 539(3)}{}

\cventry{2023}{\normalfont{\emph{Relative baryon-dark matter velocities in cosmological zoom simulations}}}{}{\textbf{Conaboy}, Iliev, Fialkov, Dixon, Sullivan}{MNRAS 525(4)}{}

\subsection{\em Coauthor}
\cventry{2025}{\normalfont{\em Square Kilometre Array Science Data Challenge 3a: Foreground removal for an EoR experiment}}{}{SKA Data Challenge participants, \textbf{Conaboy}}{MNRAS 543(2)}{}
\cventry{}{\normalfont{\em Reionization in HESTIA: Studying reionization in the LG through zoom simulations}}{}{Attard, \textbf{Conaboy}, Libeskind, Pillipenko, Dixon, Iliev}{arXiv e-prints}{}
  
\cventry{}{\normalfont{\em The Thermal Sunyaev-Zel'dovich Effect
    from the Epoch of Reionization}}{}{Iliev, Hosein, Chluba, \textbf{Conaboy}, Attard, Mondal, Ahn, Gottl\"ober, Lewis, Ocvirk, Park,
  Shapiro, Sorce, and Yepes}{MNRAS 540(2)}{}

\cventry{}{\normalfont{\em How probable is the Lyman-$\alpha$ damping
    wing in the spectrum of the redshift z = 5.9896 quasar ULAS
    J0148+0600?}}{}{Sawyer, Bolton, Becker, \textbf{Conaboy},
  Haehnelt, Keating, Kulkarni, Puchwein}{MNRAS 539(3)}{}

\cventry{2024}{\normalfont{\em Reproducibility of machine learning
    analyses of 21 cm reionization maps}}{}{Sooknunan, Chapman,
  \textbf{Conaboy}, Mortlock, and Pritchard}{arXiv e-prints}{}

  
\cventry{}{\normalfont{\em Dust-UV offsets in high-redshift
    galaxies in the Cosmic Dawn III simulation}}{}{Ocvirk, Lewis,
  \textbf{ Conaboy}, Dubois, Bethermin, Sorce, Aubert, Shapiro,
  Dawoodbhoy, Lee, Teyssier, Yepes, Gottl\"ober, Iliev, Ahn, and
  Park}{arXiv e-prints (in review)}{}

\cventry{}{\normalfont{\emph{The Lyman-limit photon mean free path
      at the end of late reionization in the Sherwood–Relics
      simulations}}}{}{Feron, \textbf{Conaboy}, Bolton, Chapman,
  Haehnelt, Keating, Kulkarni, Puchwein}{MNRAS 532(2)}{}

\cventry{2022}{\normalfont{\emph{The short ionizing photon mean free
      path at z=6 in Cosmic Dawn III, a new fully-coupled
      radiation-hydrodynamical simulation of the Epoch of
      Reionization}}}{}{Lewis, Ocvirk, Sorce, Aubert,
  \textbf{Conaboy}, Shapiro, Dawoodbhoy, Teyssier, Yepes,
  Gottl\"{o}ber, Ahn, Iliev, Th\'{e}lie}{MNRAS 516(3)}{}

  \section{Conference proceedings}
  \cventry{2022}{\normalfont{\emph{The Reionisation of the Local
        Universe in the Hestia Suite}}}{John von Neumann
    Institute for Computing Symposium}{\textbf{Conaboy},
    Iliev, Libeskind}{}{}

  \section{Supervision}
  % \cventry{2022-present}{\normalfont{Assistant supervisor to a PhD
  %     student}}{}{}{University of
  %     Nottingham}{}{}
  \subsection{Assistant supervisor}
  \cventry{2022-present}{\normalfont{Jennifer Feron}}{}{}{PhD, University of Nottingham}{}{}

  
  \section{Talks}
  \cventry{Oct 2025}{\normalfont{CCS Colloquium}}{\normalfont{CCS, Tsukuba}}{{\bfseries invited}}{}{}
  \cventry{Oct 2025}{\normalfont{KMM Discussion Meeting}}{\normalfont{NAOJ, Tokyo}}{{\bfseries invited}}{}{}
  \cventry{Sep 2025}{\normalfont{Kaba Kada EoR Meeting}}{\normalfont{Port Douglas}}{}{}{}
  \cventry{Apr 2025}{\normalfont{Cake Talk}}{\normalfont{DAWN, Copenhagen}}{}{}{}
  \cventry{Feb 2024}{\normalfont{Astronomy Colloquium}}{\normalfont{ICC, Durham}}{{\bfseries invited}}{}{}
  \cventry{Nov 2023}{\normalfont{TIFR State of the Universe}}{\normalfont{TIFR, Mumbai (online)}}{}{}{}
  \cventry{Nov 2023}{\normalfont{Cambridge Galaxies Discussion Group}}{\normalfont{KICC, Cambridge}}{{\bfseries invited}}{}{}
  \cventry{Jun 2023}{\normalfont{CLUES Meeting 2023}}{\normalfont{LMU CAS, Munich}}{}{}{}
  % \cventry{Jul 2022}{\normalfont{\emph{Hestia in the EoR}}}{}{CLUES Meeting 2022}{Madrid}{}
  \cventry{Jul 2022}{\normalfont{CLUES Meeting 2022}}{\normalfont{UAM, Madrid}}{}{}{}
  \cventry{Sep 2021}{\normalfont{RAMSES User Meeting 2021}}{\normalfont{online}}{}{}{}
  % \cventry{Sep 2021}{\normalfont{\emph{Baryon drift effects at
  %       high-redshift}}}{}{RAMSES User Meeting 2021}{online}{}

  % \cventry{Dec 2020}{\normalfont{\emph{Baryon drift effects on
  %       small-scale structure}}}{}{PhD in a Pandemic 2020}{online}{}
  \cventry{Dec 2020}{\normalfont{PhD in a Pandemic 2020}}{\normalfont{online}}{}{}{}
  
  % \cventry{Nov 2020}{\normalfont{\emph{Local Group zoom simulations
  %       and baryon drift}}}{}{CLUES Mid-term Meeting}{online}{}
  \cventry{Nov 2020}{\normalfont{CLUES Mid-term Meeting}}{\normalfont{online}}{}{}{}
  
  % \cventry{Oct 2020}{\normalfont{\emph{Baryon drift effects on small-scale
  %       structure}}}{}{Third Global 21cm Meeting}{online}{}
  \cventry{Oct 2020}{\normalfont{Third Global 21cm Meeting}}{\normalfont{online}}{}{}{}

  % \cventry{Jan 2020}{\normalfont{\emph{Cosmological
  %       reionisation}}}{}{RAS Specialist Meeting}{London, UK}{}
    \cventry{Oct 2020}{\normalfont{RAS Specialist Meeting}}{{\normalfont London, UK}}{}{}{}

    \clearpage
  \section{Grants}
  % \subsection{Summary}
    % \cventry{}{\normalfont The approximate monetary value of each compute time award
    % is estimated using the core-hour rate offered by the University of
    % Nottingham of £0.015 per core-hour (Oct 2025).}{}{}{}{}

  \cventry{}{\normalfont I have been involved in multiple successful
    applications for computing time, both at national and
    international facilities. The total monetary value of the result
    of these applications is over £1M in compute time, of which £660k
    is as Co-PI. The approximate monetary value of each compute time
    award is estimated using the core-hour rate offered by the
    University of Nottingham of £0.015 per core-hour (Oct
    2025).}{}{}{}{}
  
  \subsection{\em Co-PI}
  % \cventry{2025}{\normalfont Co-PI (PI: Dr Noam Libeskind, PC: Prof Ilian Iliev) JUWELS
  %   (J\"{u}lich, Germany) computing time  (\textbf{14.2M core-h})}{£213k}{}{}{}
  % \cventry{2024}{\normalfont Co-PI (PI: Dr Noam Libeskind, PC: Prof Ilian Iliev) JUWELS
  %   (J\"{u}lich, Germany) computing time  (\textbf{8.8M core-h})}{£132k}{}{}{}
  % \cventry{2023}{\normalfont Co-PI (PI: Dr Noam Libeskind, PC: Prof Ilian Iliev) JUWELS
  %   (J\"{u}lich, Germany) computing time  (\textbf{10.0M core-h})}{£150k}{}{}{}
  % \cventry{2022}{\normalfont Co-PI (PI: Dr Noam Libeskind, PC: Prof Ilian Iliev) JUWELS
  %   (J\"{u}lich, Germany) computing time (13.3M core-h, applied) (\textbf{11.0M core-h, granted})}{£165k}{}{}{}

  \cventry{2025}{\normalfont JUWELS (J\"{u}lich, Germany) computing time
    (PI: Dr Noam Libeskind, PC: Prof Ilian Iliev) \textbf{14.2M core-h}}{£213k}{}{}{}
  \cventry{2024}{\normalfont JUWELS (J\"{u}lich, Germany) computing time
    (PI: Dr Noam Libeskind, PC: Prof Ilian Iliev) \textbf{8.8M core-h}}{£132k}{}{}{}
  \cventry{2023}{\normalfont JUWELS (J\"{u}lich, Germany) computing time
    (PI: Dr Noam Libeskind, PC: Prof Ilian Iliev) \textbf{10.0M core-h}}{£150k}{}{}{}
  \cventry{2022}{\normalfont JUWELS (J\"{u}lich, Germany) computing time
    (PI: Dr Noam Libeskind, PC: Prof Ilian Iliev) \textbf{11.0M core-h}}{£165k}{}{}{}

  
  \subsection{\em Contributor}
  \cventry{2025}{\normalfont EuroHPC, LUMI-C (Kajani, Finland) computing time
    (PI: Dr Laura Keating) \textbf{15.1M core-h}}{£227k}{}{}{}

  \cventry{2024}{\normalfont DiRAC (Cambridge, Edinburgh, and Durham, UK) computing time
    (PI: Prof James Bolton) \textbf{12.7M core-h}}{£190k}{}{}{}

  \cventry{2021}{\normalfont PRACE, Beskow/Dardel KTH (Stockholm, Sweden) computing time (PI: Prof Ilian Iliev) \textbf{12.0M core-h}}{£180k}{}{}{}
  
  \section{Awards and press releases}
  \cventry{2022}{\normalfont Physics Finalist}{}{STEM for Britain}{}{Selected to present a poster in the Houses of Parliament to members of both Houses. Attendance covered in \href{https://www.inyourarea.co.uk/news/leicestershire-scientist-to-present-star-formation-research-in-uk-parliament}{\underline{local news media}}, including BBC radio.}
  
  % \section{Project participation}
  % \cventry{2022-present}{\normalfont Square Kilometre Array-Epoch of Reionisation (SKA-EoR) data challenge team}{}{}{}{}
  % \cventry{2018-present}{\normalfont Cosmic Dawn (CoDa) simulation team}{}{}{}{}

  \section{Conferences and workshops}
  \subsection{Organising}
  \cventry{2025}{\normalfont EAS 2025 SS22 -- Modelling the first billion years}{SOC}{Cork, Ireland}{}{Selecting talks}
  \cventry{2022--present}{\normalfont University of Nottingham Astronomy Seminar Series}{Organiser}{Nottingham, UK}{}{Inviting and hosting guest speakers}
  \cventry{2020}{\normalfont SAZERAC 2020}{LOC}{online}{}{Moderating discussion}
  \section{Teaching}
    \cventry{2020}{\normalfont Foundation Mathematics B}{University of
      Sussex}{}{}{Assisting with workshops, marking.}{}
    
  \cventry{2019}{\normalfont Scientific Computing}{University of
    Sussex}{}{}{Assisting with workshops, marking.}{}

  \cventry{2019}{\normalfont Financial Computing with
    MATLAB}{University of Sussex}{}{}{Assisting with workshops,
    marking.}
  
  \section{Computing skills}
  \cventry{}{\normalfont I am an experienced user of various programming languages and astophysical codes, those which I have used extensively are listed below.}{}{}{}{}
  \cventry{Languages}{\normalfont Python (\texttt{numpy, scipy,
      matplotlib, mpi4py, yt}), Fortran, bash/sh, MPI, C/C++, Matlab}{}{}{}{}
  
  \cventry{Codes}{\normalfont \texttt{ramses, music, camb, gadget, aton}}{}{}{}{}
  
  \cventry{Other}{\normalfont Linux, High performance computing,
    \LaTeX}{}{}{}{}

  % \subsection{Attended}
  % %\cventry{Year}{Conf}{}{}{}{}
  % % \cventry{}{\normalfont A non-exhaustive list of workshops, meetings and
  % % conferences attended}{}{}{}{}
  % \cventry{2022}{\normalfont SKA-EoR meeting}{}{}{}{}
  % \cventry{2022}{\normalfont CLUES meeting}{}{}{}{}
  % \cventry{2021}{\normalfont RAMSES User Meeting}{}{}{}{}
  % \cventry{2020}{\normalfont PhD in a Pandemic}{}{}{}{}
  % \cventry{2020}{\normalfont Third Global 21cm Meeting}{}{}{}{}
  % \cventry{2020}{\normalfont RAS Specialist Meeting}{Radiation-hydrodynamics}{}{}{}
  % \cventry{2019}{\normalfont RAMSES User Meeting}{}{}{}{}
  % \cventry{2019}{\normalfont RAS Specialist Meeting}{Machine Learning in Astronomy}{}{}{}
  % \cventry{2018}{\normalfont Parallel and GPU Programming in Python}{}{}{}{}
  % \cventry{2018}{\normalfont ICIC Data Analysis Workshop}{}{}{}{}
  % \section{Outreach}

%\clearpage
% \section{Master's thesis}
% \cvitem{Title}{\emph{Investigating the response in African climate anomalies to the El Ni\~no-Southern Oscillation}}
% \cvitem{Supervisor}{Prof Frazer Pearce}
% \cvitem{Authors}{Luke Conaboy and Mark G. Skilbeck}
% \cvitem{Description}{We explore the use of remote sensing methods in
%   examining the effects of El Ni\~no, La Ni\~na and other
%   ocean-atmosphere systems on cloud coverage and vegetation growth in
%   South and East Africa.}

%% \subsection{Vocational}
%% \cventry{year--year}{Job title}{Employer}{City}{}{General description no longer than 1--2 lines.\newline{}
%% Detailed achievements:
%% \begin{itemize}
%% \item Achievement 1
%% \item Achievement 2 (with sub-achievements)
%%   \begin{itemize}
%%   \item Sub-achievement (a);
%%   \item Sub-achievement (b), with sub-sub-achievements (don't do this!);
%%     \begin{itemize}
%%     \item Sub-sub-achievement i;
%%     \item Sub-sub-achievement ii;
%%     \item Sub-sub-achievement iii;
%%     \end{itemize}
%%   \item Sub-achievement (c);
%%   \end{itemize}
%% \item Achievement 3
%% \item Achievement 4
%% \end{itemize}}
%% \cventry{year--year}{Job title}{Employer}{City}{}{Description line 1\newline{}Description line 2\newline{}Description line 3}
%% \subsection{Miscellaneous}
%% \cventry{year--year}{Job title}{Employer}{City}{}{Description}

%% \section{Languages}
%% \cvitemwithcomment{Language 1}{Skill level}{Comment}
%% \cvitemwithcomment{Language 2}{Skill level}{Comment}
%% \cvitemwithcomment{Language 3}{Skill level}{Comment}
%% \cvitemwithcomment{Language 4}{Skill level}{Comment}

%% \section{Computer skills}
%% \cvdoubleitem{category 1}{XXX, YYY, ZZZ}{category 4}{XXX, YYY, ZZZ}
%% \cvdoubleitem{category 2}{XXX, YYY, ZZZ}{category 5}{XXX, YYY, ZZZ}
%% \cvdoubleitem{category 3}{XXX, YYY, ZZZ}{category 6}{XXX, YYY, ZZZ}

%% \section{Skill matrix}
%% \cvitem{Skill matrix}{Alternatively, provide a skill matrix to show off your skills}
%% %% Skill matrix as an alternative to rate one's skills, computer or other. 

%% %% Adjusts width of skill matrix columns. 
%% %% Usage \setcvskillcolumns[<width>][<factor>][<exp_width>]
%% %% <width>, <exp_width> should be lengths smaller than \textwidth, <factor> needs to be between 0 and 1.
%% %% Examples:
%% % \setcvskillcolumns[5em][][]%    adjust first column. Same as \setcvskillcolumns[5em]
%% % \setcvskillcolumns[][0.45][]%   adjust third (skill) column. Same as \setcvskillcolumns[][0.45]
%% % \setcvskillcolumns[][][\widthof{``Year''}]%     adjust fourth (years) column.
%% % \setcvskillcolumns[][0.45][\widthof{``Year''}]%
%% % \setcvskillcolumns[\widthof{``Languag''}][0.48][]
%% % \setcvskillcolumns[\widthof{``Languag''}]%

%% %% Adjusts width of legend columns. Usage \setcvskilllegendcolumns[<width>][<factor>]
%% %% <factor> needs to be between 0 and 1. <width> should be a length smaller than \textwidth
%% %% Examples:
%% % \setcvskilllegendcolumns[][0.45]
%% % \setcvskilllegendcolumns[\widthof{``Legend''}][0.45]
%% % \setcvskilllegendcolumns[0ex][0.46]% this is usefull for the banking style

%% %% Add a legend if you are using \cvskill{<1-5>} command or \cvskillentry
%% %% Usage \cvskilllegend[*][<post_padding>][<first_level>][<second_level>][<third_level>][<fourth_level>][<fifth_level>]{<name>}
%% % \cvskilllegend % insert default legend without lines
%% \cvskilllegend*[1em]{}% adjust post spacing
%% % \cvskilllegend*{Legend}%  Alternatively add a description string
%% %% adjust the legend entries for other languages, here German
%% % \cvskilllegend[0.2em][Grundkenntnisse][Grundkenntnisse und eigene Erfahrung in Projekten][Umfangreiche Erfahrung in Projekten][Vertiefte Expertenkenntnisse][Experte\,/\,Spezialist]{Legende}

%% %% Alternative legend style with the first three skill levels in one column
%% %% Usage \cvskillplainlegend[*][<post_padding>][<first_level>][<second_level>][<third_level>][<fourth_level>][<fifth_level>]{<name>}
%% % \setcvskilllegendcolumns[][0.6]%  works for classic, casual, banking
%% % \setcvskilllegendcolumns[][0.55]%  works better for oldstyle and fancy
%% % \cvskillplainlegend{}
%% % \cvskillplainlegend[0.2em][Grundkenntnisse][Grundkenntnisse und eigene Erfahrung in Projekten][Umfangreiche Erfahrung in Projekten][Vertiefte Expertenkenntnisse][Experte/Guru]{Legende}

%% %% Add a head of the skill matrix table with descriptions.
%% %% Usage \cvskillhead[<post_padding>][<Level>][<Skill>][<Years>][<Comment>]%
%% \cvskillhead[-0.1em]%   this inserts the standard legend in english and adjust padding
%% %% Adjust head of the skill matrix for other languages
%% % \cvskillhead[0.25em][Level][F\"ahigkeit][Jahre][Bemerkung]

%% %% \cvskillentry[*][<post_padding>]{<skill_cathegory>}{<0-5>}{<skill_name>}{<years_of_experience>}{<comment>}% 
%% %% Example usages:
%% \cvskillentry*{Language:}{3}{Python}{2}{I'm so experienced in Python and have realised a million projects. At least.}
%% \cvskillentry{}{2}{Lilypond}{14}{So much sheet music! Man, I'm the best!}
%% \cvskillentry{}{3}{\LaTeX}{14}{Clearly I rock at \LaTeX}
%% \cvskillentry*{OS:}{3}{Linux}{2}{I only use Archlinux btw}% notice the use of the starred command and the optional 
%% \cvskillentry*[1em]{Methods}{4}{SCRUM}{8}{SCRUM master for 5 years}
%% %% \cvskill{<0-5>} command
%% % \cvitem{\textbackslash{cvskill}:}{Skills can be visually expressed by the \textbackslash{cvskill} command, e.g. \cvskill{2}}

%% \section{Interests}
%% \cvitem{hobby 1}{Description}
%% \cvitem{hobby 2}{Description}
%% \cvitem{hobby 3}{Description}

%% \section{Extra 1}
%% \cvlistitem{Item 1}
%% \cvlistitem{Item 2}
%% \cvlistitem{Item 3. This item is particularly long and therefore normally spans over several lines. Did you notice the indentation when the line wraps?}

%% \section{Extra 2}
%% \cvlistdoubleitem{Item 1}{Item 4}
%% \cvlistdoubleitem{Item 2}{Item 5\cite{book2}}
%% \cvlistdoubleitem{Item 3}{Item 6. Like item 3 in the single column list before, this item is particularly long to wrap over several lines.}

%% \section{References}
%% \begin{cvcolumns}
%%   \cvcolumn{Category 1}{\begin{itemize}\item Person 1\item Person 2\item Person 3\end{itemize}}
%%   \cvcolumn{Category 2}{Amongst others:\begin{itemize}\item Person 1, and\item Person 2\end{itemize}(more upon request)}
%%   \cvcolumn[0.5]{All the rest \& some more}{\textit{That} person, and \textbf{those} also (all available upon request).}
%% \end{cvcolumns}

%% % Publications from a BibTeX file without multibib
%% %  for numerical labels: \renewcommand{\bibliographyitemlabel}{\@biblabel{\arabic{enumiv}}}% CONSIDER MERGING WITH PREAMBLE PART
%% %  to redefine the heading string ("Publications"): \renewcommand{\refname}{Articles}
%% \nocite{*}
%% \bibliographystyle{plain}
%% \bibliography{publications}                        % 'publications' is the name of a BibTeX file

%% % Publications from a BibTeX file using the multibib package
%% %\section{Publications}
%% %\nocitebook{book1,book2}
%% %\bibliographystylebook{plain}
%% %\bibliographybook{publications}                   % 'publications' is the name of a BibTeX file
%% %\nocitemisc{misc1,misc2,misc3}
%% %\bibliographystylemisc{plain}
%% %\bibliographymisc{publications}                   % 'publications' is the name of a BibTeX file

%% \clearpage
%% %-----       letter       ---------------------------------------------------------
%% % recipient data
%% \recipient{Company Recruitment team}{Company, Inc.\\123 somestreet\\some city}
%% \date{January 01, 1984}
%% \opening{Dear Sir or Madam,}
%% \closing{Yours faithfully,}
%% \enclosure[Attached]{curriculum vit\ae{}}          % use an optional argument to use a string other than "Enclosure", or redefine \enclname
%% \makelettertitle

%% Lorem ipsum dolor sit amet, consectetur adipiscing elit. Duis ullamcorper neque sit amet lectus facilisis sed luctus nisl iaculis. Vivamus at neque arcu, sed tempor quam. Curabitur pharetra tincidunt tincidunt. Morbi volutpat feugiat mauris, quis tempor neque vehicula volutpat. Duis tristique justo vel massa fermentum accumsan. Mauris ante elit, feugiat vestibulum tempor eget, eleifend ac ipsum. Donec scelerisque lobortis ipsum eu vestibulum. Pellentesque vel massa at felis accumsan rhoncus.

%% Suspendisse commodo, massa eu congue tincidunt, elit mauris pellentesque orci, cursus tempor odio nisl euismod augue. Aliquam adipiscing nibh ut odio sodales et pulvinar tortor laoreet. Mauris a accumsan ligula. Class aptent taciti sociosqu ad litora torquent per conubia nostra, per inceptos himenaeos. Suspendisse vulputate sem vehicula ipsum varius nec tempus dui dapibus. Phasellus et est urna, ut auctor erat. Sed tincidunt odio id odio aliquam mattis. Donec sapien nulla, feugiat eget adipiscing sit amet, lacinia ut dolor. Phasellus tincidunt, leo a fringilla consectetur, felis diam aliquam urna, vitae aliquet lectus orci nec velit. Vivamus dapibus varius blandit.

%% Duis sit amet magna ante, at sodales diam. Aenean consectetur porta risus et sagittis. Ut interdum, enim varius pellentesque tincidunt, magna libero sodales tortor, ut fermentum nunc metus a ante. Vivamus odio leo, tincidunt eu luctus ut, sollicitudin sit amet metus. Nunc sed orci lectus. Ut sodales magna sed velit volutpat sit amet pulvinar diam venenatis.

%% Albert Einstein discovered that $e=mc^2$ in 1905.

%% \[ e=\lim_{n \to \infty} \left(1+\frac{1}{n}\right)^n \]

%% \makeletterclosing

%\clearpage\end{CJK*}                              % if you are typesetting your resume in Chinese using CJK; the \clearpage is required for fancyhdr to work correctly with CJK, though it kills the page numbering by making \lastpage undefined
\end{document}


%% end of file `template.tex'.


%%% Local Variables:
%%% mode: LaTeX
%%% TeX-master: t
%%% End:
